\chapter{Marco teórico}

Seguramente el concepto de energía sea el más usado, conocido e importante en ciencias. A pesar de su gran empleo en la actualidad, fue desarrollado con lentitud a lo largo de los siglos y culminó con ley de la conservación de la energía. La energía se define como la capacidad de realizar un trabajo, por lo tanto, existe una gran cantidad de manifestaciones de la energía, y el calor es quizá, la manifestación de energía más común. Es importante recordar que el calor no se considera como algo almacenado dentro de un cuerpo. Al igual que el trabajo, existe sólo como energía transitoria que va de un cuerpo a otro o entre un sistema y su medio. En termodinámica, exiten varias cantidades que estan relacionadas con la enrgía, entre ellas se encuentran la enegía interna y la entalpía. \\

Podemos definir a la \textbf{energía interna} como la suma de todas las energías cinéticas y potenciales de los componentes de un sistema (una definición muy conveniente para una escala macroscópica). No obstante, si nos refirieramos a una escala microscópica, podríamos decir que, la energía interna es la energía relacionada con el movimientos aleatorio y desordenado de las moléculas, es decir, la energía interna esta formada por energías de traslación, rotación, vibración, electrónica y nuclear, asi como interacciones intermoleculares. El simbolo que representa a la energía interna  es la letra \textbf{U} y escribimos su cambio como 

\begin{equation}
\Delta U = U_{2} - U_{1}
\end{equation}

donde $U_{2}$ y $U_{1}$ son las energías internas del sistema en su estado final e inicial, respectivamente \cite{Chang2000}.

A la cantidad de energía de un sistema que se encuentra a presión constante se le conoce como \textbf{entalpía (H)} y matematicamente se define como 
\begin{equation}
 H = U + PV
\end{equation}

donde U, P y V son  la energía interna, la presión y volumen del sistema. H tiene unidades de energía.\\

Dado que U, P y V son funciones de estado, H también lo será. Por lo tanto, el cambio en la entalpía dependerá sólo de las condiciones iniciales y finales del sistema:

\begin{equation}
\Delta H = H_{2} - H_{1} = (U_{2} + P_{2}V_{2})-(U_{1} + P_{1}V_{1})
\end{equation}


\subsection{Entalpía de reacción}

Existen procesos químicos que se llevan a cabo a presión constante, un ejemplo de ello son las reacciones químicas que pueden considerarse como sistemas termodinámicos. El cambio de entalpía que acompaña a una reacción se conoce como \textbf{entalpía de reacción} o \textit{calor de reacción} y algunas veces se representa como $\Delta H_{rxn}$ \cite{Brown2014, Garritz2005}. Como cualquier sistema, el cambio en la entalpía puede ser exotérmico o endotérmico. Dicho lo anterior, en una reacción, se puede escribir el cambio en la entalpía como

\begin{equation}
\Delta H = H_{productos} - H_{reactivos}
\end{equation}

\subsection{Condiciones estándar}

Al cuantificar la entalpía, es posible determinarla a distintos valores de presión y temperatura, no obstante, los valores reportados suelen medirse a una presión de 1 atm y a una temperatura de 25 °C (298.15 K), lo anterior, es necesario para poder contar con un marco de referencia. Del mismo modo, el estado estándar de un sistema puede ser definido como la forma más estable de un elemento o de un compuesto, cuando el sistema se encuentra en condiciones estándar de temperatura y presión \cite{Garritz2005}.
