\chapter{Marco teórico}

Seguramente el concepto de energía sea el más usado, conocido e importante en ciencias. A pesar de su gran empleo en la actualidad, fue desarrollado con lentitud a lo largo de los siglos y culminó con ley de la conservación de la energía. La energía se define como la capacidad de realizar un trabajo, por lo tanto, existe una gran cantidad de manifestaciones de la energía, y el calor es quizá, la manifestación de energía más común. Es importante recordar que el calor no se considera como algo almacenado dentro de un cuerpo. Al igual que el trabajo, existe sólo como energía transitoria que va de un cuerpo a otro o entre un sistema y su medio. En termodinámica, exiten varias cantidades que estan relacionadas con la enrgía, entre ellas se encuentran la enegía interna y la entalpía. \\

Podemos definir a la \textbf{energía interna} como la suma de todas las energías cinéticas y potenciales de los componentes de un sistema (una definición muy conveniente para una escala macroscópica). No obstante, si nos refirieramos a una escala microscópica, podríamos decir que, la energía interna es la energía relacionada con el movimientos aleatorio y desordenado de las moléculas, es decir, la energía interna esta formada por energías de traslación, rotación, vibración, electrónica y nuclear, asi como interacciones intermoleculares. El simbolo que representa a la energía interna  es la letra \textbf{U} y escribimos su cambio como 

\begin{equation}
\Delta U = U_{2} - U_{1}
\end{equation}

donde $U_{2}$ y $U_{1}$ son las energías internas del sistema en su estado final e inicial, respectivamente \cite{Chang2000}.

A la cantidad de energía de un sistema que se encuentra a presión constante se le conoce como \textbf{entalpía (H)} y matematicamente se define como 
\begin{equation}
 H = U + PV
\end{equation}

donde U, P y V son  la energía interna, la presión y volumen del sistema. H tiene unidades de energía.\\

Dado que U, P y V son funciones de estado, H también lo será. Por lo tanto, el cambio en la entalpía dependerá sólo de las condiciones iniciales y finales del sistema:

\begin{equation}
\Delta H = H_{2} - H_{1} = (U_{2} + P_{2}V_{2})-(U_{1} + P_{1}V_{1})
\end{equation}


\subsection{Entalpía de reacción}

Existen procesos químicos que se llevan a cabo a presión constante, un ejemplo de ello son las reacciones químicas que pueden considerarse como sistemas termodinámicos. El cambio de entalpía que acompaña a una reacción se conoce como \textbf{entalpía de reacción} o \textit{calor de reacción} y algunas veces se representa como $\Delta H_{rxn}$ \cite{Brown2014, Garritz2005}. Como cualquier sistema, el cambio en la entalpía puede ser exotérmico o endotérmico. Dicho lo anterior, en una reacción, se puede escribir el cambio en la entalpía como

\begin{equation}
\Delta H = H_{productos} - H_{reactivos}
\end{equation}

\subsection{Condiciones estándar}

Al cuantificar la entalpía, es posible determinarla a distintos valores de presión y temperatura, no obstante, los valores reportados suelen medirse a una presión de 1 atm y a una temperatura de 25 °C (298.15 K), lo anterior, es necesario para poder contar con un marco de referencia. Del mismo modo, el estado estándar de un sistema puede ser definido como la forma más estable de un elemento o de un compuesto, cuando el sistema se encuentra en condiciones estándar de temperatura y presión \cite{Garritz2005}.

\subsection{Calorimetría}

Es posible determinar en forma experimental el valor de $\Delta H$, esto se obtiene al medir el flujo de calor (cambio de temperatura). A esta medición se le conoce como \textbf{calorimetría}; el calorímetro es un dispositivo empleado para medir el flujo de calor. Las técnicas y el equipo empleados en calorimetría dependen de la naturaleza del proceso en estudio. Para muchas reacciones, como las que ocurren en disolución, es fácil controlar la presión y medir directamente el $\Delta H$. Aunque los calorímetros utilizados para trabajos muy minuciosos son instrumentos de precisión \cite{Brown2014}.


\subsection{Ley de Hess}

Con frecuencia, se calcula el $\Delta H$ de una reacción a partir de los valores tabulados de otras reacciones. Por ello, no es necesario realizar mediciones calorimétricas para todas las reacciones. Como la entalpía es una función de estado, el cambio de entalpía asociado con cualquier proceso químico sólo depende de la cantidad de materia que experimenta el cambio, y de la naturaleza del estado inicial de los reactivos y del estado final de los productos. Esto significa que si una reacción se lleva a cabo en una o en varias etapas, la suma de los cambios de entalpía asociados con las etapas individuales debe ser igual al cambio de entalpía asociado con el proceso de una sola etapa \cite{Brown2014}.

La \textbf{ley de Hess} establece que si una reacción se realiza en una serie de etapas, el $\Delta H$ de la reacción completa, será igual a la suma de los cambios de entalpía en las etapas individuales. El cambio total de entalpía del proceso es independiente del número de etapas y de la trayectoria que siga la reacción. Esta ley es una consecuencia del hecho de que la entalpía es una función de estado. Por lo tanto, se puede calcular el $\Delta H$ de cualquier proceso siempre y cuando se encuentre una trayectoria para la cual se conozca $\Delta H$ para cada etapa. Esto significa que un número relativamente pequeño de mediciones experimentales permite calcular el $\Delta H$ de un gran número de reacciones. La ley de Hess aporta un medio útil para calcular cambios de energía que son difíciles de medir directamente \cite{Brown2014}.

\subsection{Entalpía de formación}

Los métodos recién analizados permiten calcular los cambios de entalpía para una gran cantidad de reacciones, a partir de los valores tabulados de $\Delta H$. Por ejemplo, hay extensas tablas de entalpías de vaporización ($\Delta H$ para convertir líquidos a gases), entalpías de fusión ($\Delta H$ para fusión de sólidos), entalpías de combustión ($\Delta H$ para la combustión de una sustancia en oxígeno), y así sucesivamente. Un proceso particularmente importante empleado para tabular datos termoquímicos es la formación de un compuesto a partir de sus elementos constitutivos. El cambio de entalpía asociado con este proceso se conoce como \textbf{entalpía de
formación} o \textit{calor de formación}, $\Delta H_{f}$, donde el subíndice f indica que la sustancia se formó a partir de sus elementos constitutivos. La magnitud de cualquier cambio de entalpía depende de la temperatura, la presión y el estado (gas, líquido o forma sólida cristalina) de los reactivos y productos. Para comparar las entalpías de diferentes reacciones, debe definirse un conjunto de condiciones, conocido como estado estándar, en el que se tabulan la mayoría de las entalpías. El estado estándar de una sustancia es su forma pura a presión atmosférica (1 atm) y la temperatura de interés, que generalmente se elige como 298.15 K (25 °C). El \textbf{cambio de entalpía estándar} de una reacción se define como el cambio de entalpía cuando todos los reactivos y productos se encuentran en sus estados estándar. El cambio de entalpía estándar se denota con $\Delta H^{\circ}$, donde el superíndice $\circ$ indica condiciones de estado estándar \cite{Brown2014}.
La \textbf{entalpía estándar de formación} de un compuesto, $\Delta H_{f}^{\circ}$, es el cambio de entalpía de una reacción que forma un mol del compuesto a partir de sus elementos, con todas las sustancias en sus estados estándar:\\

\begin{center}
elementos (en estado estándar) $\longrightarrow$ compuesto (1 mol en estado estándar)
\end{center}

Por lo regular, se reportan valores de $\Delta H_{f}^{\circ}$ a T = 298.15 K. Si un elemento existe en más de una forma en condiciones estándar, la forma más estable del elemento es la que normalmente se utiliza para la reacción de formación.

\begin{equation}
 2C(grafito) + 3H_{2}(g) + \frac{1}{2}O_{2}(g) \longrightarrow C_{2}H_{5}OH(l)
\label{eq:3.5}
\end{equation}


La fuente elemental de oxígeno es $O_{2}$, no O ni $O_{3}$, porque el $O_{2}$ es la forma estable del oxígeno a T = 298.15 K y a presión atmosférica. De forma similar, la fuente elemental del carbono es el grafito y no el diamante, porque el grafito es la forma más estable (menor energía) a T = 298.15 K y a presión atmosférica. De igual manera, la forma más estable del hidrógeno en condiciones estándar es el $H_{2}$(g), así que este se utiliza como la fuente del hidrógeno en la ecuación \ref{eq:3.5}.


Por definición, la e\textit{ntalpía estándar de formación} de la forma más estable de cualquier elemento es cero, porque no se necesita una reacción de formación cuando el elemento ya se encuentra en su estado estándar. Así, los valores de $\Delta H_{f}^{\circ}$ para el $C$(grafito), $H_{2}$(g),$O_{2}$(g) y los estados estándar de otros elementos son cero \cite{Brown2014}.


%Leer el chang en página 102 (Raymong Chang FQ-204)
