\chapter{Marco teórico}

Seguramente el concepto de energía sea el más usado, conocido e importante en ciencias. A pesar de su gran empleo en la actualidad, 
fue desarrollado con lentitud a lo largo de los siglos y culminó con ley de la conservación de la energía. La energía se define como la  
capacidad de realizar un trabajo, por lo tanto, existe una gran cantidad de manifestaciones de la energía, y el calor es quizá, la manifestación 
de energía más común. Es importante recordar que el calor no se considera como algo almacenado dentro de un cuerpo. Al igual que el trabajo, 
existe sólo como energía transitoria que va de un cuerpo a otro o entre un sistema y su medio. En termodinámica, la entalpía es una propiedad 
que depende de la cantidad de energía de un sistema. 



La energía es una de las propiedades más importantes en química y puede ser obtenida con cierta facilidad empleando cálculos de estructura electrónica, algunas de esas magnitudes termodinámicas son:
\begin{enumerate}
\item Entalpia de formación.
\item Entropía.
\item Energía libre de Gibbs.
\end{enumerate}
Estas magnitudes son fundamentales porque brindan información acerca de la estabilidad y la termodinámica de las moléculas. A partir de ello, es posible entender los efectos que tienen los procesos químicos. Teóricamente, es posible obtener dichas magnitudes haciendo uso de tablas (valores determinados de forma experimental) o mediante cálculos \textit{ab initio}. Esta última opción es valiosa porque (1) es mucho más sencillo y económico que hacer un experimento termoquímico, (2) existen compuestos que no han sido medidos ni tabulados y (3) hay compuestos que son altamente reactivos, o compuestos de interés biológico que están disponibles sólo en pequeñas cantidades, por lo tanto, no es posible someterlos a rígidos protocolos experimentales, \textit{v.gr}. reacciones de combustión \cite{Lewars2016}. Durante los años sesenta, Jhon Pople y sus colaboradores, desarollaron métodos computaciones que facilitaron un gran número de complejos cálculos en química cuántica, este esfuerzo se vió culminado en el desarollo del software \textbf{Gaussian}. Una categoria de cálculos muy populares de este software son las denominadas teorías gaussian-n y se usan para determinar valores de energías de sistemas moleculares que contienen átomos desde el hidrógeno hasta el cloro. Algunos de esos métodos son: Gn (G1,G2, G3, G4).  

 Para el caso de la entalpia de formación a 0 K, existen dos enfoques principales que utilizan la teoría Gn. Uno de ellos se basa en el método de atomización y el otro emplea el método de formación. Sin embargo, las energías obtenidas por estos métodos a una temperatura de 0 K, no son de mucha utilidad. Para calcular propiedades termodinámicas a una temperatura de 298 K, es necesario hacer correcciones térmicas de la energía interna. La mécanica estadística permite determinar valores de energía a 298 K al separar la función de partición en un producto de sus componentes traslacionales, rotacionales, vibracionales, electrónicos y nucleares. En dichas componentes existen aportaciones de (3/2) RT para la contribución traslacional, (3/2) RT para la contribución rotacional (RT para moléculas lineales) y un RT adicional para convertir la energía en entalpía (el llamado término PV). También, es utilizada la aproximación del oscilador armónico para el componente vibracional, en el que existe una expresión analítica \cite{McQuarrie1976}. Las contribuciones de los términos electrónicos y nucleares son ignoradas (es decir, las funciones de partición correspondientes se establecen en la unidad). No obstante, estos procedimientos suelen ser demasiado tediosos cuando se examinan de forma manual (por la exagerada cantidad de expresiones algebráicas que deben ser evaluadas de forma continua). Por lo tanto, es necesario contar con una herramienta computacional que permita optimizar el tiempo que requieren estos cálculos y es precisamente el objetivo de este trabajo. 


