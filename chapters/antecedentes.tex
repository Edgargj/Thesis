\chapter{Marco teórico}

\section{Energía}
Seguramente el concepto de energía sea el más usado, conocido e importante en ciencias. A pesar de su gran empleo en la actualidad, fue desarrollado con lentitud a lo largo de los siglos y culminó con ley de la conservación de la energía. La energía se define como la capacidad de realizar un trabajo, por lo tanto, existe una gran cantidad de manifestaciones de la energía, y el calor es quizá, la manifestación de energía más común. Es importante recordar que el calor no se considera como algo almacenado dentro de un cuerpo. Al igual que el trabajo, existe sólo como energía transitoria que va de un cuerpo a otro o entre un sistema y su medio. En termodinámica, exiten varias cantidades que estan relacionadas con la energía, entre ellas se encuentran la enegía interna y la entalpía. Podemos definir a la \textbf{energía interna} como la suma de todas las energías cinéticas y potenciales de los componentes de un sistema (una definición muy conveniente para una escala macroscópica). No obstante, si nos refirieramos a una escala microscópica, podríamos decir que, la energía interna es la energía relacionada con el movimientos aleatorio y desordenado de las moléculas, es decir, la energía interna esta formada por energías de traslación, rotación, vibración, electrónica y nuclear, asi como interacciones intermoleculares. El simbolo que representa a la energía interna  es la letra \textbf{U} y escribimos su cambio como 

\begin{equation}
\Delta U = U_{2} - U_{1}
\end{equation}

donde $U_{2}$ y $U_{1}$ son las energías internas del sistema en su estado final e inicial, respectivamente \cite{Chang2008}.

El estudio de la variación de la energía interna en las transformaciones químicas tiene un gran significado para el desarrollo de las bases teóricas de la química.
Además, la variación de la energía interna, es una magnitud indispensable para los cálculos termodinámicos de las reacciones químicas, porque tiene un gran significado para las investigaciones científicas y en la industria. En el proceso de transformación de una sustancia, la energía interna , se produce como en otros casos, mediante la absorción o desprendimiento de calor y la realización de trabajo. El calor de una reacción química frecuentemente es apreciable; éste, puede ser medido directamente en muchos casos y es precisamente, el objetivo de la \textbf{Termoquímica}, estudiar el calor de las reacciones químicas \cite{Guerasimov1971}.\\


A la cantidad de energía de un sistema que se encuentra a presión constante se le conoce como \textbf{entalpía (H)} y matematicamente se define como 
\begin{equation}
 H = U + PV
\end{equation}

donde U, P y V son  la energía interna, la presión y volumen del sistema. H tiene unidades de energía.\\

Dado que U, P y V son funciones de estado, H también lo será. Por lo tanto, el cambio en la entalpía dependerá sólo de las condiciones iniciales y finales del sistema:

\begin{equation}
\Delta H = H_{2} - H_{1} = (U_{2} + P_{2}V_{2})-(U_{1} + P_{1}V_{1})
\end{equation}

\section{Entalpía de reacción}

Existen procesos químicos que se llevan a cabo a presión constante, un ejemplo de ello son las reacciones químicas que pueden considerarse como sistemas termodinámicos. El cambio de entalpía que acompaña a una reacción se conoce como \textbf{entalpía de reacción} o \textit{calor de reacción} y algunas veces se representa como $\Delta H_{rxn}$ \cite{Brown2014, Garritz2005}. Como cualquier sistema, el cambio en la entalpía puede ser exotérmico o endotérmico. Dicho lo anterior, en una reacción, se puede escribir el cambio en la entalpía como

\begin{equation}
\Delta H = H_{productos} - H_{reactivos}
\end{equation}

\section{Calorimetría}

Es posible determinar en forma experimental el valor de $\Delta H$, esto se obtiene al medir el flujo de calor (cambio de temperatura). A esta medición se le conoce como \textbf{calorimetría}; el calorímetro es un dispositivo empleado para medir el flujo de calor. Las técnicas y el equipo empleados en calorimetría dependen de la naturaleza del proceso en estudio. Para muchas reacciones, como las que ocurren en disolución, es fácil controlar la presión y medir directamente el $\Delta H$. Aunque los calorímetros utilizados para trabajos muy minuciosos son instrumentos de precisión \cite{Brown2014}.

\section{Entalpía de formación estándar}

Un proceso particularmente importante empleado para tabular datos termoquímicos es la formación de un compuesto a partir de sus elementos constitutivos. El cambio de entalpía asociado con este proceso se conoce como \textbf{entalpía de formación} o \textit{calor de formación}, $\Delta H_{f}$, donde el subíndice f indica que la sustancia se formó a partir de sus elementos constitutivos. Sus unidades son  $kJ$ (mol de reacción)$^{-1}$. La magnitud de cualquier cambio de entalpía depende de la temperatura, la presión y el estado (gas, líquido o forma sólida cristalina) de los reactivos y productos. Para comparar las entalpías de diferentes reacciones, debe definirse un conjunto de condiciones, conocido como estado estándar, en el que se tabulan la mayoría de las entalpías. El estado estándar de una sustancia es su forma pura a presión atmosférica (1 atm) y la temperatura de interés, que generalmente se elige como 298.15 K (25 °C). El \textbf{cambio de entalpía estándar} de una reacción se define como el cambio de entalpía cuando todos los reactivos y productos se encuentran en sus estados estándar. El cambio de entalpía estándar se denota con $\Delta H^{\circ}$, donde el superíndice $\circ$ indica condiciones de estado estándar \cite{Brown2014}.
La \textbf{entalpía estándar de formación} de un compuesto, $\Delta H_{f298}^{\circ}$, es el cambio de entalpía de una reacción que forma un mol del compuesto a partir de sus elementos, con todas las sustancias en sus estados estándar:

\begin{center}
elementos (en estado estándar) $\longrightarrow$ compuesto (1 mol en estado estándar)
\end{center}

Por lo regular, se reportan valores de $\Delta H_{f}^{\circ}$ a T = 298.15 K. Si un elemento existe en más de una forma en condiciones estándar, la forma más estable del elemento es la que normalmente se utiliza para la reacción de formación.

\begin{equation}
 2C(grafito) + 3H_{2}(g) + \frac{1}{2}O_{2}(g) \longrightarrow C_{2}H_{5}OH(l)
\label{eq:3.5}
\end{equation}


La fuente elemental de oxígeno es $O_{2}$, no O ni $O_{3}$, porque el $O_{2}$ es la forma estable del oxígeno a T = 298.15 K y a presión atmosférica. De forma similar, la fuente elemental del carbono es el grafito y no el diamante, porque el grafito es la forma más estable a T = 298.15 K y a P = 1 atm. De igual manera, la forma más estable del hidrógeno en condiciones estándar es el $H_{2}$(g), así que este se utiliza como la fuente del hidrógeno en la ecuación \ref{eq:3.5}.


Por definición, la \textit{entalpía estándar de formación} de cualquier elemento es cero, porque no se necesita una reacción de formación cuando el elemento ya se encuentra en su estado estándar. Así, los valores de $\Delta H_{f}^{\circ}$ para el $C$(grafito), $H_{2}$(g),$O_{2}$(g) y los estados estándar de otros elementos son cero \cite{Brown2014, Chang2008}.

\begin{equation}
\Delta H_{f}^{\circ} (grafito) = 0
\label{eq:3.6}
\end{equation}

\begin{equation}
\Delta H_{f}^{\circ} (H_{2}) = 0
\label{eq:3.7}
\end{equation}

\begin{equation}
\Delta H_{f}^{\circ} (O_{2}) = 0
\label{eq:3.8}
\end{equation}

Como se mecionó antes, no podemos determinar el valor absoluto de la entalpía de una sustancia. Sólo se pueden calcular valores relativos a una referencía arbitraria. En termodinámica, lo que nos interesa principalmente, son los cambio de $H$, aunque todo valor de $\Delta H_{f}^{\circ}$ asignado arbitrariamente a un elemento funcionaría, con ceros los cálculos se simplifican \cite{Chang2008}. Los valores de $\Delta H_{f}^{\circ}$ pueden obtenerse con un método directo o indirecto, a continuación se describen.

\textbf{Método directo}. Este método funciona en compuestos que se pueden sintetizar con facilidad a partir de sus elementos. Por ejemplo, la formación de $CO_{2}$ a partir de grafito y $O_{2}$.

\textbf{Método indirecto}. La mayor parte de los compuestos no se pueden sintetizar directamente a partir de sus elementos. En algunos casos, las reacciones suceden con demasida lentitud o no llegan a concretarse. Por lo tanto, el valor de $\Delta H_{f}^{\circ}$ se puede determinar con un método indirecto basado en la ley de Hess \cite{Chang2008}. 

\section{Ley de Hess}

Con frecuencia, se calcula el $\Delta H$ de una reacción a partir de los valores tabulados de otras reacciones. Por ello, no es necesario realizar mediciones calorimétricas para todas las reacciones. Como la entalpía es una función de estado, el cambio de entalpía asociado con cualquier proceso químico sólo depende de la cantidad de materia que experimenta el cambio, y de la naturaleza del estado inicial de los reactivos y del estado final de los productos. Esto significa que si una reacción se lleva a cabo en una o en varias etapas, la suma de los cambios de entalpía asociados con las etapas individuales debe ser igual al cambio de entalpía asociado con el proceso de una sola etapa.\\

La \textbf{ley de Hess} establece que si una reacción se realiza en una serie de etapas, el $\Delta H$ de la reacción completa, será igual a la suma de los cambios de entalpía en las etapas individuales. El cambio total de entalpía del proceso es independiente del número de etapas y de la trayectoria que siga la reacción. Esta ley es una consecuencia del hecho de que la entalpía es una función de estado. Por lo tanto, se puede calcular el $\Delta H$ de cualquier proceso siempre y cuando se encuentre una trayectoria para la cual se conozca $\Delta H$ para cada etapa. Esto significa que un número relativamente pequeño de mediciones experimentales permite calcular el $\Delta H$ de un gran número de reacciones. La ley de Hess aporta un medio útil para calcular cambios de energía que son difíciles de medir directamente 
\cite{Brown2014}. 

Sería imposible medir de forma directa la entalpía en el caso de la reacción de combustión de carbono para formar monóxido de carbono (la combustión de 1 mol de $C$ con 0.5 moles de $O_{2}$ produce tanto $CO$ como $CO_{2}$, dejando algún $C$ sin reaccionar). Sin embargo, el carbono sólido y el monóxido de carbono pueden quemarse por completo en $0_{2}$ para producidir $CO_{2}$. Por lo tanto, es posible utilizar los cambios de entalpía de estas reacciones para calcular el calor de combustión del carbono. Sí se observa, la reacción general es la formación de $CO_{2}$ (reacción \ref{eq:3.9}) que se puede descomponer en dos partes (reaciones \ref{eq:3.10} y \ref{eq:3.11}). La figura \ref{CO} muestra el esquema general de nuestro procedimiento.

\begin{equation}
\begin{aligned}
& C(grafito) + O_{2}(g) \longrightarrow CO_{2}(g) & \Delta H_{f}^{\circ} = -393.5 kJ\cdot mol^{-1}
\end{aligned}
\label{eq:3.9}
\end{equation}

\begin{equation}
\begin{aligned}
& CO(g) + \frac{1}{2} O_{2}(g) \longrightarrow CO_{2}(g) & \Delta H_{f}^{\circ} = -283.0 kJ\cdot mol^{-1}
\end{aligned}
\label{eq:3.10}
\end{equation}

\begin{equation}
\begin{aligned}
& CO_{2}(g) \longrightarrow CO(g) + \frac{1}{2} O_{2}(g) & \Delta H_{f}^{\circ} = 283.0 kJ\cdot mol^{-1}
\end{aligned}
\label{eq:3.11}
\end{equation}

\begin{equation}
\begin{aligned}
& C(grafito) + \frac{1}{2} O_{2}(g) \longrightarrow CO_{2}(g) & \Delta H_{f}^{\circ} = -110.5 kJ\cdot mol^{-1}
\end{aligned}
\label{eq:3.12}
\end{equation}

\begin{figure}[hbtp]
\begin{center}
\includegraphics[scale=1]{graphs/CO.png}
\caption{Diagrama de entalpía para la formación de monóxido de carbono. El cambio de entalpía de la reacción total es igual a la suma de los cambios de entalpía de los dos pasos.}
\label{CO}
\end{center}
\end{figure}

\section{Química computacional}
Los métodos recién analizados permiten calcular los cambios de entalpía para una gran cantidad de reacciones. Por ejemplo, hay extensas tablas de entalpías de vaporización, entalpías de fusión y entalpías de combustión \cite{NIST1998, Tajti2004, Nicolaides1996}. No obstante, también es posible cuantificar la entalpía de formación mediante cálculos \textit{ab initio} \cite{Lewars2016}.\\

La determinación de entalpías de formación por métodos computacionales es fundamental porque sirve como información de entrada para multitud de simulaciones numéricas. Tal es la importancia, que se ha invertido un gran esfuerzo en la tabulación y refinamiento de \textit{JANAF, CODATA}, \textit{Third Millenium y Active Thermochemical Tables (ATcT)}\cite{Cox1989, Goos1998, Ruscic2004, Ruscic2005, Ruscic2005b}. Aunque la determinación experimental (principalmente calorimetría) es el método ideal para obtener $\Delta H_{f}^{\circ}$, es a la vez, laboriosa y consume mucho tiempo. Es claro que obtener $\Delta H_{f}^{\circ}$ experimental para todas los compuestos existentes es imposible y costoso. No obstante, ahora se acepta que la química teórica, ya sea de forma aislada o en conjunto con el experimento, pueda ofrecer un medio eficaz para obtener esta información en plazos factibles.\\

La búsqueda del ``cielo químico'' en química computacional esta sujeto en gran medida, al tipo de metodología que sea capaz de predecir calores de formación con desviaciones menores a $1 kcal\cdot mol^{-1}$ respecto al valor experimental. Por lo tanto, la precisión de cualquier $\Delta H_{f}^{\circ}$ depende del modelo químico aplicado y del tamaño del sistema molecular. Los métodos compuestos ofrecen una alternativa rentable a expensas de la precisión, consisten en una serie de optimizaciones en la geometría y cálculos de energía en un solo punto, con correcciones empíricas para superar las deficiencias cuando los métodos son comparados con entalpías de formación conocidas. Númerosos métodos se han usado en los últimos tiempos; los métodos CBS-x \cite{Montgomery2000, Ochterski1996} (x = QBB3 y APNO) y Gaussian-x \cite{Pople1989,Curtiss1990, Curtiss1991, Curtiss1998, Curtiss2007} (x = 1, 2, 3, 4), que son empleados en cálculos cinéticos y termodinámicos, siendo de gran relevancia para la termoquímica. 

Variantes del método G3, tales como G3B3 \cite{Baboul1999} y G3MP2 \cite{Curtiss1999} también son populares, y el primero se emplea con frecuencia para la $\Delta H_{f}^{\circ}$ reportada por Burcat y otros autores, como parte de la base de datos \textit{Third Millenium}, con una incertidumbre de $8 kJ \cdot mol^{-1}$. Los métodos ``estándar'' se han ajustado en numerosos ocasiones para remediar las dificultades percibidas con el objetivo de mejorar el coste computacional;  por ejemplo, variantes  de G4 \cite{Chan2010, Chan2011}, W1 \cite{Chan2012, Karton2012} y W3 \cite{Chan2013, Gruzman2009}. Para superar las grandes incertidumbres y errores en $\Delta H_{f}^{\circ}$  se emplea el método isodésmico, sin embargo, surge  un gran número de problemas al emplear este método porque se requieren entalpías de formación experimentales o métodos computacionales con un alto nivel de teoría para crear estas reacciones de trabajo hipotéticas;  los $\Delta H_{f}^{\circ}$ calculados depende de la calidad de la reacción isodésmica (dado que no hay una única respuesta), además, el costo computacional se incrementa bastante. Pese a esto, los métodos computacionales pueden ofrecer excelentes resultados.

\section{Métodos Gaussianos}

La clave de estos métodos es el uso de altos niveles de correlación y grandes conjuntos de bases. Esta serie comenzó en 1989 con Gaussian 1, G1 \cite{Pople1989}, continuó con G2 (1991) \cite{Curtiss1991} y G3 (1998) \cite{Curtiss1998}, y vio la publicación (2007) de G4 \cite{Curtiss2007}. G1 y G2 están obsoletos. Los métodos Gaussianos de alta precisión más populares en la actualidad son probablemente G4 y G3 y sus variantes más rápidas, pero casi tan precisas, G4(MP2) \cite{Curtiss2007a} y G3(MP2) \cite{Curtiss1999}. El uso continuado de G3 y G3(MP2) (más bien G4 y G4(MP2)) puede estar justificado por el deseo de comparar algunos trabajos actuales con el conjunto de resultados acumulados con estos métodos algo más antiguos). Para G3, la desviación absoluta media del experimento es 1,13 kcal mol$^{-1}$ [4,7 kJ mol$^{-1}$] y para
G3(MP2) 1,2 – 1,3 kcal mol$^{-1}$ (5,0 – 5,4 kJ mol$^{-1}$), y G3(MP2) parece ser 7-8 veces más rápido que G3 \cite{Curtiss1998}. Curtis y otros autores proporcionan los detalles del método G4 \cite{Curtiss2007} y lo compararon con G3, en cierta medida, con G1 y G2. Ellos informaron que ``[…] la desviación absoluta promedio del experimento muestra una mejora significativa de 1,13 kcal/mol [4,7 kJ mol$^{-1}$] (teoría G3) a 0,83 kcal/mol [3,5 kJ mol$^{-1}$] (teoría G4). G4 es de 2 a 3 veces más lento que G3. Para acelerar el método G4, sus pasos MP4 fueron reemplazados por MP2 y MP3 dando como resultado G4(MP2) y G4(MP3) \cite{Curtiss2007a}. Estos tienen, respectivamente, desviaciones absolutas medias del experimento de 1,04 kcal/mol [4,35 kJ mol$^{-1}$] y 1,03 kcal/mol [4,3 kJ mol$^{-1}$]. El método G4(MP2) parece en general ser el mejor
de los dos; es de 2 a 3 veces más rápido que G3 y, aunque aproximadamente el doble de lento que G3 (MP2), Curtiss y otros autores mencionaron \cite{Curtiss2007a}.``En general, el método G4 (MP2) proporciona un método preciso y económico para las predicciones termodinámicas''. Tiene una precisión general para el conjunto de pruebas de moléculas G3/05 que es significativamente mejor que la teoría G3(MP2) (1,04 frente a 1,39 kcal/mol) [4,35 frente a 5,8 kJ mol$^{-1}$] e incluso mejor que la teoría G3 (1,04 frente a 1,13 kcal/mol) [4,35 frente a 4,7 kJ mol$^{-1}$]. G4(MP2) y G3(MP2) pueden manejar moléculas con hasta 16 átomos pesados. Debido a que el método G4 es mucho más rápido y preciso sus pasos se resumirán aquí. Un cálculo G4, utiliza los siguientes pasos

Estos siete pasos se utilizan para ensamblar una energía molecular como la suma de varias diferencias de energía y un incremento de energía empírico final (la ``corrección de nivel superior’’) en función del número de electrones apareados y no apareados. La energía G4(MP2) es esencialmente un tipo de energía CCSD(T)/6–31G* realizada en un B3LYP/6–31G (2df, p), con una ZPE escalada B3LYP/6–31G (2df, p) y una corrección de energía empírica, pero un cálculo tan directo sería más lento que dividirlo en los pasos utilizados aquí. Una forma en que G4 (MP2) ahorra tiempo es reemplazando un MP3 y un cálculo de MP4 con un MP2. Una mejora clave en G4/G4(MP2) sobre G3/G3(MP2) es el reemplazo del método de correlación cuadrática CI por el
método de conglomerados acoplados; este cambio en particular no alteró la precisión del
conjunto de pruebas de moléculas, pero presumiblemente mejora la confiabilidad, como ``[…] el método QCISD(T) tiene fallas bastante dramáticas, lo que no ocurre con el método CCSD(T)’’ \cite{Curtiss2007}. Véase también Hrusak y otros autores. Para una comparación de CI cuadrático y clúster acoplado [103]. En el método G3(MP2), el principal cambio con respecto a G3 es que los cálculos de MP2 reemplazan a los de MP4 \cite{Curtiss1998}. \\ 
Debido a las correcciones de energía empírica en los métodos multipasos gaussianos, no son completamente ab initio, sino algo semiempíricos, excepto cuando estas correcciones se anulan. Esta cancelación ocurre, por ejemplo, al calcular las afinidades de los protones como la diferencia de energía de las especies protonadas y no protonadas, donde las correcciones de la órbita de espín y el número de electrones de espín $\alpha$ y $\beta$ son los mismos en ambos lados de la ecuación. Tomaremos G4 como el método Gaussiano de elección, con un buen compromiso entre precisión y velocidad, pero también nos referiremos a los cálculos G3(MP2), debido a su velocidad y precisión competitivas y su extenso legado. Estos métodos gaussianos, y otros no mencionados aquí, así como los métodos CBS son revisados (2012) por Peterson, Feller y Dixon, con énfasis en
termoquímica, estructuras y frecuencias; estos autores reconocen que 4 kJ mol$^{-1}$ (1 kJ mol$^{-1}$) como precisión química aceptada en termoquímica [185]. Aquí hay una comparación de los métodos gaussianos y un método CBS, usando 1,4-benzoquinona (p-benzoquinona, $O=C_{6}H_{4}=O$), mostrando una mejor precisión y tiempo para el caso de G4(MP2) al calcular la entalpía de formación por el método de atomización, se debe tener en cuenta que el cálculo G4 tomó nueve veces más tiempo (160/18) que el G4 (MP2) y ambos dieron casi la misma entalpía de formación: \\ 

\begin{table}
\begin{center}
\begin{tabular}{||l|c|c||}
\hline 
Entalpía de formación & kJ mol$^{-1}$ & minutos/tiempo relativo \\ 
\hline 
G4 & -117.5 & 160/21 \\ 
\hline 
G4(MP2) & -118.5 & 18/2.4 \\ 
\hline 
G3 & -118.6 & 24/3.2 \\ 
\hline 
G3(MP2) & -120.0 & 7.5/1 \\ 
\hline 
CBS-QB3 & -115.9 & 16/2.1 \\ 
\hline 
\end{tabular} 
\end{center}
\caption{La entalpía de formación aceptada en la literatura es de -122. $\pm$ 3.8 kJ mol$^{1}$ [186]. Los primeros cuatro valores calculados experimentalmente están dentro de 1 kJ mol$^{1}$, sin embargo, el valor CBS-QB3 es 3 kJ mol$^{-1}$ por encima del mayor error estimado. No obstante, no debemos generalizar a partir de una muestra de un solo compuesto.}
\label{Ximello-table}
\end{table}

\section{Métodos CBS}
La clave de estos métodos es la extrapolación del conjunto de bases a un límite infinito (hasta el final). Hay tres métodos CBS básicos: CBS-4 (para extrapolación de cuarto orden), CBS-Q (para CI cuadrático) y CBS-APNO (para orbitales naturales de pares asintóticos, referidos a la extrapolación al límite establecido de base), en orden de aumentando la precisión (y aumentando el tiempo de computadora) [113]. Estos métodos están disponibles con palabras clave en los programas Gaussian 94 y Gaussian posteriores, donde las versiones preferidas de CBS-4 y CBS-Q se especifican mediante las palabras clave CBS-4M [187a] y CBS-QB3 [187b] (M para localización de población mínima, B3 para el uso del funcional de densidad B3LYP). CBS-4M puede manejar
moléculas con hasta 19 átomos pesados y tiene sus “mayores errores en los calores neutros de formación para ClF3 (13,6 kcal/mol), O3 (12,6 kcal/mol) y C2Cl4 (11,0 kcal/mol)”, pero “estos errores son sistemáticos y su efecto puede reducirse en gran medida mediante el uso de correcciones de aditividad de enlaces isodésmicos”. [187a]. Los errores CBS-4M más típicos son (desviación absoluta media del experimento) 3,26 kcal mol"1 (13,6 kJ mol"1) [187a]. Hay una modificación de CBS-4M diseñada para disminuir la acumulación de errores con el aumento del tamaño molecular [188]. CBS-QB3 puede manejar moléculas con hasta 13 átomos pesados y tiene una desviación media absoluta del experimento de 1,10 kcal mol"1 (4,6 kJ mol"1) [187b]. CBS-APNO puede manejar moléculas con hasta aproximadamente 7 átomos pesados y tiene una desviación absoluta media del experimento de 0,53 kcal mol"1 (2,2 kJ mol"1) [113]. Consulte la Tabla 5.10. Los métodos de conjuntos básicos completos [113] implican esencialmente siete u ocho pasos:

\begin{enumerate}
\item Una optimización de geometría (en el nivel HF/3–21G(*) o MP2/6–31G*, dependiendo del método CBS en particular).
\item Un cálculo de ZPE a nivel de optimización.
\item Un cálculo de punto único de HF con un conjunto de bases muy grande (6–311 + G(3d2f,2df,p) o 6–311 + G(3d2f,2df,2p), dependiendo del método CBS en particular).
\item Un cálculo de punto único MP2 (base que depende del método CBS particular).
\item Algo llamado extrapolación de un par de orbitales naturales para estimar el error debido al uso de un conjunto de bases finitas.
\item Un cálculo de punto único MP4.
\end{enumerate}

\section{Comparacion entre métodos} 

\begin{comment}
Otro trabajo de Bond también mostró poca diferencia entre las entalpías de formación por reacciones de tipo isodésmico de los métodos G3 y G3 (MP2) [193]. Ess y Houk encontraron que CBS-QB3 era satisfactorio para las entalpías de activación de las reacciones pericíclicas [194], lo cual es notable porque los métodos de alta precisión que estamos discutiendo fueron diseñados para dar buenos resultados para la termodinámica, no para la cinética; el problema aquí radica en la parametrización, particularmente para espines apareados y no apareados, cuyo número podría alterarse a lo largo de una coordenada de reacción [195]. Sin embargo, se ha declarado explícitamente que CBS-QB3 es adecuado para energías de activación [187b]. En la Tabla 5.10 se proporciona una indicación de las capacidades de velocidad y tamaño de G4(MP2), G3(MP2) y CBS-4M, CBS-QB3 y CBS-APNO.
\end{comment}



