\chapter{Conclusión}

El trabajo realizado cumplió el principal objetivo planteado desde un inicio, esto es, la creación de tres programas de cómputo científico que cálculan entalpías de formación de compuestos orgánicos a T = 298.15 K con correciones en la energía interna utilizando aproximaciones tales como rotor rigido - oscilador armónico y Nicolaides, empleando archivos de salida de Gaussian09 que incorpora valores de métodos con un alto nivel de teoría y conjuntos de bases pequeñas, los cuales fueron: G3, G3MP2, G4, CBS-APNO y CBS-QB3. De igual forma, se añadieron valores experimentales reportados por la comunidad científica (NIST, Tajti y Argonne) que permiten comparar los resultados con otros datos base.

Adicionalmente, se lograron otros objetivos específicos:

• El manejo del programa se realiza a través de la línea de comandos.
• No requiere la intervención del usuario para funcionar correctamente.
• La salida del programa es muy simple, un letrero con el resultado de la comparación.
• El código es OpenSource y se encuentra disponible en un repositorio de GitHub.
Es posible utilizar este programa en diferentes ámbitos de la química computacional,
brindando una herramienta rápida y fácil de usar, para mejorar los flujos de trabajo.


