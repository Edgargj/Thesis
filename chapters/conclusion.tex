\chapter{Conclusión}

El trabajo realizado cumplió con el objetivo principal, es decir; la creación de tres programas de cómputo científico que calculan entalpías de formación de compuestos orgánicos a T = 298.15 K con correciones en la energía interna utilizando aproximaciones como Nicolaides y otros, rotor rigido y oscilador armónico, empleando archivos de salida de Gaussian09. Además, se incorporaron valores de métodos con un alto nivel de teoría y conjuntos de bases pequeñas, los cuales fueron: G3, G3MP2, G4, CBS-APNO y CBS-QB3. De igual forma, se añadieron valores experimentales reportados por la comunidad científica: Tajti y otros, NIST y Argonne. Que permiten comparar los resultados con otros datos base.\\

También, se lograron otros objetivos específicos:

\begin{itemize}
\item Los programas fueron diseñados para utilizarse a través de una línea de comandos en un sistema operativo de GNU/Linux, dando como resultado, una alta eficiencia en el flujo de trabajo.

\item Se Implementó una programación orientada a objetos que fragmentó el código en partes independientes, permitiendo así, reciclar el código para futuros proyectos.
\end{itemize}





