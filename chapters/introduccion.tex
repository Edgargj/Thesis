\chapter{Introducción}

Los ordenadores son máquinas potentes que podemos utilizar para resolver problemas de la vida cotidiana, sin embargo, su propósito original fue la implementación de funciones aritméticas en los circuitos electrónicos que contiene para, posteriormente, configurarse y hacer cálculos algebráicos en ella. Este fue uno de los grandes logros de la comunidad científica en el siglo XX. Actualmente, su uso en las ciencias es fundamental y el desarrollo del cómputo ha creado una nueva rama de la química, la química computacional. Es posible calcular geometrías moleculares, equilibrios de reacciones, espectros y otras propiedades físicas con las herramientas de esta nueva rama. Algunas veces un experimento puede ser considerado muy peligros o incluso, imposible de ser desarrollado en el laboratorio (puede resultar muy costoso o requerir demasiado tiempo). Existen compuestos que son muy reactivos para ser aislados y por eso, son incapaces de ser estudiados por métodos comunes de laboratorio. Este tipo de moléculas pueden estudiarse con métodos computacionales, desde luego, no debe considerase a la química computacional como una rival de las técnicas experimentales tradicionales, ambas logran resultados que son imposibles de obtener de forma individual. Por esta razón, la química computacional no reemplaza al experimento, lo complementa. La sorprendente evolución de las matemáticas, la física y la química teórica han contribuido al florecimiento de la química computacional, al proveernos de conceptos, modelos teóricos, métodos numéricos y analíticos más eficientes que se han incorporado en algoritmos programables\cite{Cuevas2003}. La química computacional es una disciplina que comprende todos aquellos aspectos de la investigación en química que se benefician de la aplicación de los ordenadores. 
Las simulaciones efectuadas por computadoras tienen varias ventajas:
\begin{enumerate}
\item Son más económicas y rápidas que los experimentos físicos.
\item Pueden resolver un amplio margen de problemas que lo que se podrían resolver con equipos de laboratario específicos o de tecnología actual.
\item Los cálculos sólo están limitados por la velocidad del ordenador y la capacidad de memoria de éste.
\end{enumerate}

La química computacional puede dividirse en dos categorías; métodos de mecánica molecular y métodos de mecánica cuántica. Los métodos de mecánica molecular (MM) se fundamentan en las leyes de la mecánica clásica (consideran a los átomos como partículas puntuales dotadas de masa y carga, unidos por enlaces que pueden ser tratados como resortes). \\

Por otro lado, los métodos que tienen como fundamento la mecánica cuántica, emplean la ecuación de Schrödinger para describir una molécula con un tratamiento directo de la estructura electrónica abordada a través de métodos \textit{ab initio} (significa “desde el principio” y se refiere a que en este tipo de cálculos se emplean constantes fundamentales de la física) y métodos  \textit{semiempíricos} (emplea parámetros cuyos valores se ajustan con datos experimentales de cálculos \textit{ab initio}). Los métodos \textit{ab initio} y los métodos \textit{semiempíricos}, se enfocan en las predicciones de las propiedades de los sistemas atómicos y moleculares. No existe un método "mejor" que otro porque todo dependerá del sistema a analizar, los recursos de cómputo disponibles y la precisión requerida. Adicionalmente, existen dos factores importantes para elegir un método de cálculo adecuado; la naturaleza de la molécula y los parámetros conocidos de la molécula\cite{Cuevas2003}. \\

La energía es una de las propiedades más importantes que se calcula en la determinación de la estructura electrónica de una molécula, existen cantidades relevantes que pueden ser calculadas con cierta facilidad empleando cálculos de estructura electrónica, algunas de ellas son:
\begin{enumerate}
\item Entalpia de formación
\item Entropia 
\item Energía libre de Gibbs 
\end{enumerate}
Estas magnitudes son fundamental porque brindan información acerca de la estabilidad y la termodinámica. A partir de ello, es posible entender los efectos que tienen los procesos químicos. Cuando en una reacción se obtiene 1 mol de un compuesto a partir de sus elementos en su forma más estable, a la energía involucrada en la formación de calor se le conoce como \textbf{entalpía de formación} y se representa como $\Delta H_{f}^{\circ}$. Una forma de determinar la $\Delta H_{f}^{\circ}$ es calcinar un compuesto dentro de una bomba calorimétrica y cuantificar el cambio de temperatura para calcular la cantidad de calor involucrado en esa reacción. Teóricamente, también es posible obtener la entalpía de formación haciendo uso de tablas (existen extensas tabulaciones de entalpías de formación determinadas experimentalmente\cite{NIST1998, Tajti2004}) o mediante cálculos \textit{ab initio} \cite{Lewars2016}. 
Esta última opción es valiosa porque (1) es mucho más sencillo y económico que hacer un experimento termoquímico, (2) existen compuestos que no han sido medidos ni tabulados y (3) hay compuestos que son altamente reactivos, o compuestos de interés biológico que están disponibles sólo en pequeñas cantidades, por lo tanto, no es posible someterlos a rígidos protocolos experimentales, \textit{v.gr.} reacciones de combustión \cite{Lewars2016}. 
Existen dos enfoques principales que utilizan la teoría Gn para predecir las entalpías de formación a T = 0 K. Uno de ellos se basa en el método de atomización y el otro emplea el método de formación. Sin embargo, las energías obtenidas por estos métodos a una temperatura de T = 0 K, no son de mucha utilidad. \\

Para calcular propiedades como la entalpía, la capacidad calorífica y la entropía a T = 298 K, es necesario hacer correcciones térmicas de la energía interna. La Termodinámica Estadística permite obtener valores de energía a T = 298 K al separar la función de partición en un producto de sus componentes traslacionales, rotacionales, vibracionales, electrónicos y nucleares\cite{McQuarrie1976, Nicolaides1996}. En dichas componentes existen aportaciones de (3/2) RT para la contribución traslacional, (3/2) RT para la contribución rotacional (RT para moléculas lineales) y un RT adicional para convertir la energía en entalpía (el llamado término PV). La aproximación del oscilador armónico es utilizada para el componente vibracional, en el que existe una expresión analítica. Las contribuciones de los términos electrónicos y nucleares son ignorados (es decir, las funciones de partición correspondientes se establecen en la unidad). No obstante, estos procedimientos suelen ser demasiado tediosos cuando se examinan de forma manual (por la exagerada cantidad de expresiones algebráicas que deben ser evaluadas de forma continua). Por lo tanto, es necesario contar con una herramienta que permita optimizar el tiempo que requieren estos cálculos.\\

 La precisión de un cálculo de energía varía notablemente con el nivel de teoría y con la base de cálculo. Afortunadamente, existen metodologías que permiten determinar la energía con una precisión de $\pm$ 2kJ/mol, respecto a una determinación experimental. Estos métodos constan de secuencias de cálculos predefinidos y son combinados para lograr valores muy precisos con costos computacionales aceptables, fueron introducidos por Pople y otros autores \cite{Cuevas2003}. Una categoría muy popular son los denominados teorías gaussian-n y se usan para el cálculo de energías de sistemas moleculares que contienen átomos desde el hidrógeno hasta el de cloro, con objetivo de desarrollar procedimientos generales, de amplia aplicabilidad para cualquier molécula capaz de reproducir valores termoquímicos experimentales con la precisión antes mencionada. Algunos de esos métodos son: Gn (G1, G2, G3, G4). \\
 
Existen otras técnicas como CBS-N (CBS-4 y CBS-Q) \cite{Simmie2015}, pero en este trabajo abordaremos exclusivamente los método Gn. En términos prácticos hay cuatro pasos generales en estos procedimientos:
\begin{enumerate}
	\item Selección de la geometría molecular para efectuar un cálculo a bajo nivel (Hartree-Fock o B3LYP).
	\item Selección de una base para efectuar el cálculo.
	\item Determinación de la energía de correlación electrónica.
	\item Determinación de las constantes rotacionales, traslacionales y vibracionales.
\end{enumerate}
La teoría G4 es un procedimiento en el que se realiza una secuencia de cálculos sustentados en la teoría de orbitales moleculares  (\textit{ab initio}) y ha sido muy empleada en el cálculo de energías de enlace, entalpías de formación, potenciales de ionización y afinidades electrónicas \cite{Cuevas2003, Tajti2004}. Los pasos en un cálculo en la teoría G4 son los siguientes:
\begin{enumerate}
	\item Obtención de la estructura de equilibrio en un nivel B3LYP/6-31G(2\textit{df,p}).
	\item Determinación de las frecuencias armónicas.
	\item Determinación de límite de energía Hartree-Fock.
	\item Corrección de la energía.
	\item Determinación de la energía MP4/6-31G(\textit{d}), correcciones del paso anterior y combinación aditiva con el paso 3.
	\item Corrección de alto nivel (HLC) con parámetros empíricos.
	\item Se obtiene la energía total a T = 0 K con una corrección en la energía de punto zero obtenida en el paso 2. 
\end{enumerate}
Estos siete pasos se utilizan para ensamblar una energía molecular como la suma de varias diferencias de energía y un incremento empírico de energía final (la ``corrección de nivel superior") basado en el número de electrones apareados y no apareados\cite{Curtiss2007}.En relación con todo lo descrito anteriormente, es necesario un algoritmo de cómputo que determine la entalpia de formación de compuestos orgánicos a T = 298 K, a partir de archivos de salida del software \textit{Gaussian}.
